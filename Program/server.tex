% -*- coding: UTF8 -*-
% server.tex
% 关于服务器配置的笔记
% 使用XelaTex编译

\documentclass[UTF8]{ctexart}
\usepackage{geometry}
\usepackage{amsmath}
\usepackage{minted}


\geometry{a4paper, centering, scale=0.8}

\newtheorem{shellcommand}{shell命令}
\newenvironment{myquote}
  {\begin{quote} \kaishu \zihao{-5}}
  {\end{quote}}

\title{\heiti 服务器配置笔记}
\author{\kaishu 操杰朋}
\date{2015年9月25日}

\begin{document}
\maketitle
\clearpage

\tableofcontents
\clearpage

\section{Nginx}
\subsection{Nginx的启动,停止及平滑重启}
启动Nginx可执行Shell命令\eqref{eq:start},如不加参数“-c”,则加载默认目录下的nginx.conf。

\begin{shellcommand}
  nginx -c nginx.conf \label{eq:start}
\end{shellcommand}

Nginx的停止方法有多种,我们可以通过shell命令\eqref{eq:nginxPid}来查找Nginx的主进程号,再使用kill来结束进程。

\begin{shellcommand}
  ps -ef | grep nginx \label{eq:nginxPid}
\end{shellcommand}

Nginx的平滑重启,可以通过shell命令\eqref{eq:checkNginx}来查看Nginx配置是否正确,在使用shell命令\eqref{eq:restart}来平滑重启
\begin{shellcommand}
  nginx -t -c nginx.conf \label{eq:checkNginx}
\end{shellcommand}
\begin{shellcommand}
  nginx -s reload \label{eq:restart}
\end{shellcommand}

\subsection{Nginx配置}
以下是一个Nginx的配置文件\footnote{cat /proc/sys/fs/file-max或ulimit -n查看文件描述符数量}:

\begin{minted}{php}
# 使用的用户和组
user www www;
# 指定工作衍生的进程数(一般等于cpu的总数或其倍数)
worker_processes cpu_number;
# 指定错误日志存放路径,可选级别[ debug | info | notice | warn | error | crit]
error_log error_path error_level;
# 指定pid存放路径
pid pid_path;
# 指定文件描述符数量(worker_rlimit_nofile number设置描述符数量)
worker_rlimit_nofile number;

events
{
  #使用的网络I/O模型,Linux使用epoll,FreeBSD使用Kqueu
  use epoll;
  # 允许的连接数
  worker_connections number;
}

http
{
  include mime.types;
  default_type application/octet-stream;

  ...

  # 客户端上传大小
  client_max_body_size 8m;

  ...
  server
  {
    ...
  }
}
\end{minted}

其中server\{...\}为虚拟主机,其重要参数为:

\begin{itemize}
  \item listen ip:port:监听ip及端口,如果只写端口,表示监听所有ip
  \item server\_name [ip|domain]:监听域名或端口
  \item log\_format name format [format ...]:设置日志格式
  \item access\_log path [format]:设置日志文件
  \item expires 30d:设置缓存时间
  \item proxy\_pass url:设置反向代理
  \item proxy\_set\_header:设置代理头
\end{itemize}

\subsection{Nginx Rewrite规则}
rewrite \^{}/(.*)\$ http://news.missevan.cn/\$1 permanent;

一共有四种不同的flag,分别是redirect,permanent,break,last。前两种是跳转,后两种是代理。
\begin{itemize}
  \item redirect:302\footnote{临时跳转}跳转
  \item permanent:301\footnote{永久跳转}跳转
  \item last:会继续执行下面的location
  \item break:直接执行server
\end{itemize}

其他一些关键字:
\begin{itemize}
  \item -f:用来判断文件是否存在
  \item -d:用来判断文件夹是否存在
  \item -e:用来判断文件,文件夹是否存在
  \item -x:用来判断文件是否可以执行
  \item $\sim$:区分大小写匹配
  \item $\sim$*:不区分大小写匹配
  \item alias:在该目录下查找
  \item root:认为该目录为root,进行查找
\end{itemize}

\subsection{Nginx变量}
set \$key value;
\clearpage

\section{Apache}
\end{document} 