% -*- coding: UTF8 -*-
% server.tex
% 关于服务器配置的笔记
% 使用XelaTex编译

\documentclass[UTF8]{ctexart}
\usepackage{geometry}
\usepackage{amsmath}
\usepackage{minted}


\geometry{a4paper, centering, scale=0.8}

\newtheorem{shellcommand}{shell命令}
\newenvironment{myquote}
  {\begin{quote} \kaishu \zihao{-5}}
  {\end{quote}}

\title{\heiti 服务器配置笔记}
\author{\kaishu 操杰朋}
\date{创建:2015年9月25日 \\
更新:\today}


\begin{document}
\maketitle
\clearpage

\tableofcontents
\clearpage

\section{Nginx}
\subsection{Nginx的一些基础操作}
\begin{shellcommand}[检查配置文件]
  nginx -t -c nginx.conf
\end{shellcommand}

\begin{shellcommand}[启动Nginx]
  nginx -c nginx.conf \footnote{不加参数-c,则加载默认目录下的nginx.conf}
\end{shellcommand}

\begin{shellcommand}[快速关闭Nginx]
  nginx -s stop \footnote{可能存在不安全, 类似pkill -9 nginx或nginx -TERM nginx\_pid}
\end{shellcommand}

\begin{shellcommand}[平滑关闭Nginx]
  nginx -s quit \footnote{安全, 类似kill -QUIT nginx\_pid}
\end{shellcommand}

\begin{shellcommand}[重启Nginx]
  nginx -s reload \footnote{重新载入配置文件}
\end{shellcommand}

\begin{shellcommand}[重启Nginx]
  nginx -s reopen \footnote{使用新的日志路径,在迁移日志时使用}
\end{shellcommand}


\subsection{一个Nginx配置文件}
\begin{myquote}
  \begin{minted}{php}
# 使用的用户和组
user www www;

# 指定工作衍生的进程数(一般等于cpu的总数或其倍数)
worker_processes cpu_number;

# 指定错误日志存放路径,可选级别[ debug | info | notice
# | warn | error | crit]
error_log error_path error_level;

# 指定pid存放路径
pid pid_path;

# 指定文件描述符数量(cat /proc/sys/fs/file-max 或
# ulimit -n查看文件描述符数量)
worker_rlimit_nofile number;

events
{
  # 使用的网络I/O模型,Linux 使用epoll,FreeBSD使用Kqueu
  use epoll;

  # 每个进程允许的最多连接数
  # worker_connections不能超过worker_rlimit_nofile
  worker_connections number;
}

http
{
  # 文件扩展名与文件类型映射表
  include mime.types;

  # 默认文件类型
  default_type application/octet-stream;

  # 默认字符集,一般不设置,使用html中的meta设置
  #charset utf-8

  # mime的hash表大小
  server_names_hash_bucket_size 64;

  # http头的最大长度,超过会返回400
  client_header_buffer_size 16k;

  # 获取http最大数及大小,超过会返回414。default为8k
  large_client_header_buffers 4 32k;

  # request body的最大值,超过会返回413。 默认值为1m。
  # 设置0则为无限大。
  client_max_body_size 1m;

  # 是否调用linux senfile函数,默认为off
  sendfile on;

  # 数据包最大时,一次发送。只有当sendfile打开时有用,
  # 与tcp_nodelay互斥,默认为off。
  tcp_nopush off;

  # 长连接超时时间
  keepalive_timeout 120;

  # 只有当keep-alive打开时有用,小块数据直接发送
  # 默认为off
  tcp_nodelay off;

  # 开启目录列表访问,默认为off;
  autoindex off;

  #########FastCGI相关参数#########
  # nginx与fastcgi建立连接的时间,最长不大于75s,默认60s
  fastcgi_connect_timeout 60s;

  # nginx向fastcgi进程发送request的连续写时间,默认60s
  fastcgi_send_timeout 300s;

  # nginx读取fastcgi进程返回的response 的连续读时间,默认60s
  fastcgi_read_timeout 300s;

  fastcgi_buffer_size 64k;
  fastcgi_buffers 4 64k;
  fastcgi_busy_buffers_size 128k;
  fastcgi_temp_file_write_size 128k;

  #########gzip模块设置#########
  #开启gzip压缩输出
  gzip on;

  #最小压缩文件大小
  gzip_min_length 1k;

  #压缩缓冲区
  gzip_buffers 4 16k;

  #压缩版本(默认1.1,前端如果是squid2.5 请使用1.0)
  gzip_http_version 1.0;

  #压缩等级
  gzip_comp_level 2;

  #压缩类型,默认就已经包含text/html,所以下面就不用再写了,写上去也不会有问题,但是会有一个warn。
  gzip_types text/plain application/x-javascript text/css application/xml;

  #########虚拟主机配置########
  server
  {
    # 监听ip及端口,如果只写端口,表示监听所有ip
    listen 80;

    #监听域名或ip
    server_name www.tomcao.com 192.168.1.1;

    #设置日志
    log_format format_name '$remote_addr - $remote_user [$time_local] '
                           '"$request" $status $bytes_sent '
                           '"$http_referer" "$http_user_agent" "$gzip_ratio"';
    access_log log_path format_name buffer=32k;

    index index.html index.htm index.php;
    root /data/tomcao;

    location ^(js|css|image)
    {
      # 缓存一天
      expires 1d;
    }

    location ^/anotherweb/
    {
      proxy_pass http://anotherweb;
      proxy_set_header X-Real-IP $remote_addr;
      proxy_set_header X-Forwarded-For $proxy_add_x_forwarded_for;
    }

    location ~ \.php$
    {
      fastcgi_pass 127.0.0.1:9000;
      # fastcgi_pass unix:/dev/shm/php.sock;
      include fastcgi_params;
      fastcgi_index index.php;
    }

    location ~/\.ht
    {
      deny all;
    }
  }
}
  \end{minted}
\end{myquote}
其中fastcgi\_params详见附录。


\subsection{Nginx配置文件的操作}
设置变量使用以下命令:set \$key value。内置变量有:
\begin{myquote}
\$arg\_\emph{name}: argument \emph{name} in the request line \\
\$args: arguments in the request line \\
\$query\_string: same as \$args \\
\$cookie\_\emph{name}: cookie \emph{name} \\
\$connection: connection serial number \\
\$connection\_requests: current number of requests made through a connection\footnote{chrome会发送两次request,有一次为favicon.ico} \\
\$document\_root: root or alias directive's value for the current request \\
\$document\_uri: same as \$uri \\
\$request\_filename: file path for the current request, based on the root or alias

\end{myquote}

url重写方法:rewrite \^{}/(.*)\$ http://news.missevan.cn/\$1 flag,一共有四种不同的flag,分别是redirect,permanent,break,last。前两种是跳转,后两种是代理。
\begin{itemize}
  \item redirect:302\footnote{临时跳转}跳转
  \item permanent:301\footnote{永久跳转}跳转
  \item last:会继续执行下面的location
  \item break:直接执行server
\end{itemize}

其他一些关键字:
\begin{itemize}
  \item -f:用来判断文件是否存在
  \item -d:用来判断文件夹是否存在
  \item -e:用来判断文件,文件夹是否存在
  \item -x:用来判断文件是否可以执行
  \item $\sim$:区分大小写匹配
  \item $\sim$*:不区分大小写匹配
\end{itemize}

alias直接映射:
\begin{myquote}
  Syntax: alias path; \\
  Context: location;
\end{myquote}

root将root映射:
\begin{myquote}
  Syntax: root path; \\
  Default: root html;
  Context: http, server, location, if in location
\end{myquote}

autoindex是否生成目录:
\begin{myquote}
  Syntax: autoindex on | off; \\
  Default: autoindex off; \\
  Context: http, server, location
\end{myquote}

try\_files尝试文件是否存在,否则执行后面:
\begin{myquote}
  Syntax: try\_files file... uri; \\
  Context: server, location
\end{myquote}

\clearpage


\section{Apache}
\subsection{Apache的一些基本操作}
以下为windows下的操作:
\begin{shellcommand}[直接启动]
  httped.exe
\end{shellcommand}

\begin{shellcommand}[直接关闭]
  Control-C
\end{shellcommand}

\begin{shellcommand}[安装服务]
  httped.exe -k install -n "MyServiceName" -f httpd.conf\footnote{-f为指定配置文件}
\end{shellcommand}

\begin{shellcommand}[删除服务]
  httped.exe -k uninstall -n "MyServiceName"
\end{shellcommand}

\begin{shellcommand}[使用服务启动]
  httped.exe -k start -n "MyServiceName"
\end{shellcommand}

\begin{shellcommand}[使用服务停止1]
  httped.exe -k stop -n "MyServiceName"
\end{shellcommand}

\begin{shellcommand}[使用服务停止2]
  httped.exe -k shutdown -n "MyServiceName"
\end{shellcommand}

\begin{shellcommand}[使用服务重启]
  httped.exe -k restart -n "MyServiceName"
\end{shellcommand}

\subsection{Apache配置文件的操作}
url重写需要使用模块modules/mod\_rewrite.so\footnote{使用语句为LoadModule rewrite\_module modules/mod\_rewrite.so}。在需要使用url重写的地方使用AllowOverride  All,这样Apache可以读取目录下的.htaccess文件。使用Require all granted授予访问权限。
.htaccess的一个例子:
\begin{myquote}
  \begin{minted}{php}
# 判断重写模块是否存在
<IfModule rewrite_module>
    # FollowSymLinks必须启动,这是rewrite引擎的安全需求。
    Options +FollowSymLinks
    # hide files from apache directory listings
    IndexIgnore */*
    # 启动重写引擎
    RewriteEngine On

    # RewriteCond为重写条件,-f表示常规文件,-d表示常规目录
    RewriteCond %{REQUEST_FILENAME} !-f
    RewriteCond %{REQUEST_FILENAME} !-d
    # 使用重写的地址
    RewriteBase /myapp/
    # RewriteRule Pattern Substitution 
    RewriteRule . /index.php
</IfModule>
  \end{minted}
\end{myquote}

虚拟主机模块为核心模块组建,下面是一个简单的例子:
\begin{myquote}
  \begin{minted}{php}
# *表示对应多个ip
<VirtualHost ip:80>
  # ServerAdmin email-address 如果有什么错误,则发送邮件到该邮箱
  ServerAdmin webmaster@host.example.com
  DocumentRoot "/www/docs/host.example.com"
  ServerName host.example.com
  ErrorLog "logs/host.example.com-error_log"
  TransferLog "logs/host.example.com-access_log"
</VirtualHost>
  \end{minted}
\end{myquote}
\clearpage


\section{附录}
\subsection{fastcgi\_params}
\begin{myquote}
  \begin{minted}{php}
#脚本请求路径
fastcgi_param  SCRIPT_FILENAME    $document_root$fastcgi_script_name;
#REQUEST的query参数
fastcgi_param  QUERY_STRING       $query_string;
#REQUEST的动作(GET, POST)
fastcgi_param  REQUEST_METHOD     $request_method;
#REQUEST头中的Content-Type字段
fastcgi_param  CONTENT_TYPE       $content_type;
#REQUEST头中的Content-length字段
fastcgi_param  CONTENT_LENGTH     $content_length;

#脚本名称
fastcgi_param  SCRIPT_NAME        $fastcgi_script_name;
#请求的地址不带参数
fastcgi_param  REQUEST_URI        $request_uri;
#与uri相同
fastcgi_param  DOCUMENT_URI       $document_uri;
#网站的根目录,在server配置中root的值
fastcgi_param  DOCUMENT_ROOT      $document_root;
#请求使用的协议
fastcgi_param  SERVER_PROTOCOL    $server_protocol;
#HTTP或HTTPS,通常是HTTP/1.0或HTTP/1.1。
fastcgi_param  REQUEST_SCHEME     $scheme;
fastcgi_param  HTTPS              $https if_not_empty;

#CGI版本号
fastcgi_param  GATEWAY_INTERFACE  CGI/1.1;
#nginx版本号、可修改、隐藏
fastcgi_param  SERVER_SOFTWARE    nginx/$nginx_version;

#客户端IP
fastcgi_param  REMOTE_ADDR        $remote_addr;
#客户端端口
fastcgi_param  REMOTE_PORT        $remote_port;
#服务器IP地址
fastcgi_param  SERVER_ADDR        $server_addr;
#服务器端口
fastcgi_param  SERVER_PORT        $server_port;
#服务器名,域名在server配置中指定的server_name
fastcgi_param  SERVER_NAME        $server_name;

# PHP only, required if PHP was built with --enable-force-cgi-redirect
fastcgi_param  REDIRECT_STATUS    200;
  \end{minted}
\end{myquote}
\end{document} 